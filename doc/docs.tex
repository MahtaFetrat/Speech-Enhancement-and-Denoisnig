\documentclass[fleqn]{report}
\usepackage[utf8]{inputenc}
\usepackage{graphicx}
\usepackage{fancyhdr}
\usepackage{amsmath}
\usepackage{xcolor}
\usepackage{subfigure}
\usepackage{pdfpages}
\usepackage{hyperref}
\usepackage{xepersian}
%In order to write persian in latex, with "texlive" distro installed, two additional packages are needed: texlive-xetex & texlive-lang-arabic

\settextfont[BoldFont={XB Zar bold.ttf}]{XB Zar.ttf}

\begin{document}

%Title Page
\begin{titlepage}
    \begin{center}
        \includegraphics{sharif.png}\\
        \vspace{1cm}    \textbf{\LARGE   \lr{Speech Enhancement and Denoising}}\\
        \vspace{1cm}  \textbf{\Large  پروژه پایانی درس پردازش گفتار}\\
        \vspace{1cm}    \textbf{\large  دانشکده مهندسی کامپیوتر}\\
        \vspace{0.5cm}  \textbf{\large  استاد: دکتر صامتی}\\
        \vspace{0.5cm}  \textbf{\large  بهار ۱۴۰۳}\\
        \vspace{0.5cm}  \noindent\rule{8cm}{0.4pt}\\
        \vspace{2cm}    \textbf{\large بهار بهزادی‌پور \ \   ۴۰۲۲۰۷۱۵۵}\\
        \vspace{0.5cm}    \textbf{\large مهتا فطرت \ \   ۴۰۲۲۱۲۲۲۵}\\
    \end{center}
\end{titlepage}

\tableofcontents

%Header & Footer
\pagestyle{fancy}
\fancyhf{}
\rhead{\includegraphics[scale=0.1]{sharif.png} پردازش گفتار}
\lhead{Speech Enhancement and Denoising}
\rfoot{Page \thepage}

%Document Body

\chapter{مقدمه}
بهبود کیفیت گفتار و رفع نویز آن همواره مورد توجه بوده‌است و جزء معدود مسائل حوزه‌ی پردازش گفتار است که می‌توان گفت هنوز کاملا حل‌شده تلقی نمی‌شو و جای پیشرفت دارد.
با این حال، مدل‌ها و ابزار‌های بسیاری برای بهبود گفتار زبان انگلیسی وجود دارد که از کیفیت مطلوبی نیز برخوردار هستند.
در این میان، زبان فارسی نسبت له تکنولوژی‌های اخیر در این زمینه جای پیشرفت زیادی دارد.
بررسی‌های اولیه نشان می‌دهد که مدل بهبود گفتاری که برای زبان فارسی مناسب‌سازی شده باشد در حال حاضر وجود ندارد.
در این پروژه ما در تلاش هستیم تا وضعیت ابزار‌های حاضر را برای بهبود گفتار فارسی بررسی کنیم و بتوانیم در راه بهبود آن‌ها گام‌های موثری برداریم.

\section{شرح مسئله}
بهبود گفتار \lr{(Speech Enhancement)}  حذف نویز \lr{(Denoising)} به  فرآیندهایی اطلاق میشود که کیفیت و وضوح سیگنال گفتار را بهبود میبخشند و نویزهای پس زمینه را کاهش میدهند. این مسئله در کاربردهای مختلفی از جمله سیستم های تشخیص گفتار، ارتباطات تلفنی، و کمک به افراد با مشکلات شنوایی اهمیت دارد. در زبان فارسی، به دلیل تفاوت های صوتی و ساختاری با دیگر زبان ها، نیاز به تحقیق و توسعه ویژه ای در این زمینه وجود دارد.

\section{رویکردها}
در مورد مسئله‌ی بهبود گفتار و رفع نویز مانند بسیاری از مسائل دیگر در حوزه‌ی گفتار، دو رویکرد کلی وجود دارد.
اولین رویکرد مربوط به روش‌های 
rule-based می‌باشد.
این روش‌ها عموما مبتنی بر تکنیک‌های پردازش سیگنال می‌باشند و به مشاهده‌ی نمونه‌ی گفتار‌های تمیز و نویزی وابسته نیستند.
رویکرد دوم اما
مبتنی بر شبکه‌های 
عصبی می‌باشد.
در این روش‌ها، با داشتن یک معماری مناسب و تعداد زیادی از نمونه‌های ورودی و خروجی مطلوب، مدل می‌آموزد که تسک بهبود گفتار را انجام دهد.
در این پروژه، ما هر دوی این روش‌ها را بررسی می‌کنیم و راه‌حل‌هایی بر اساس هر یک ارائه می‌دهیم.

\section{چالش‌ها}
همانطور که برای سایر راه‌حل‌های مبتنی بر شبکه‌های عصبی  
چالش داده مطرح است، در این‌جا هم با این مشکل مواجه هستیم.
درواقع تهیه‌ی دادگانی طبیعی متشکل از زوج‌های تمیز و نویزی یک گفتار واحد، امری دشوار و زمان‌بر است. 
به همین جهت غالبا شاهد دیتاست‌هایی هستیم که به صورت اتوماتیک generate شده‌اند.
در ادامه به این موضوع بیشتر پرداخته می‌شود.

\section{اهداف و دستاورد‌های این پروژه}
در این پروژه ما ابتدا روش‌های موجود را برای زبان فارسی ارزیابی کردیم و امکان خاص‌سازی آن‌ها برای زبان فارسی را بررسی نمودیم.
سپس از بین روش‌های موجود، سعی در بهبود برخی از این روش‌ها به کمک تکنیک‌هایی چون fine-tuning داشتیم.
همچنین به عنوان یک جایگزین، ابزاری rule-based برای بهبود گفتار فارسی نیز ارائه دادیم که حتی بدون داده‌های آموزش نیز قابل استفاده است.
لازم به ذکر است که در این پروژه، دو نوع دیتاست نیز برای تسک \lr{Speech Enhancement and Denoising} برای زبان فارسی ارائه می‌شود که \lr{to the best of our knowledge} اولین دیتاست‌های این زبان در این تسک،‌ حتی از نوع ساختگی آن می‌باشد.

\chapter{مرور کارهای پیشین}
در این بخش به بررسی برخی از مهم‌ترین راه‌حل‌های موجود برای \lr{Speech Enhancement and Denoising} می‌پردازیم.

\section{Metricgan}

تفاوت بین تابع هزینه‌ای که برای آموزش مدل بهبود گفتار استفاده می‌شود و درک شنیداری انسان معمولاً باعث می‌شود که کیفیت گفتار بهبود یافته رضایت‌بخش نباشد. معیارهای ارزیابی عینی که درک انسان را در نظر می‌گیرند می‌توانند به عنوان پلی برای کاهش این فاصله عمل کنند. مدل MetricGAN که قبلاً پیشنهاد شده بود، برای بهینه‌سازی معیارهای عینی با اتصال معیار به یک تفکیک‌کننده طراحی شده بود. از آنجا که در طول آموزش تنها به امتیازات توابع ارزیابی هدف نیاز است، معیارها حتی می‌توانند غیرقابل تفکیک باشند. در این مطالعه، ما MetricGAN+ را پیشنهاد می‌کنیم که در آن سه تکنیک آموزشی که دانش حوزه پردازش گفتار را در خود دارند، پیشنهاد شده است. با این تکنیک‌ها، نتایج آزمایشی بر روی مجموعه داده VoiceBank-DEMAND نشان می‌دهد که MetricGAN+ می‌تواند امتیاز PESQ را نسبت به مدل قبلی MetricGAN به میزان 0.3 افزایش دهد و به نتایج برتر دست یابد (امتیاز PESQ = 3.15).
ایده اصلی MetricGAN شبیه‌سازی رفتار یک تابع ارزیابی هدف (مثلاً تابع PESQ) با یک شبکه عصبی (مثلاً Quality-Net [20]) است. تابع ارزیابی جانشین از امتیازات خام یاد گرفته می‌شود و تابع ارزیابی هدف را به عنوان یک جعبه سیاه در نظر می‌گیرد. هنگامی که ارزیابی جانشین آموزش داده شد، می‌توان از آن به عنوان یک تابع هزینه برای مدل بهبود گفتار استفاده کرد. متأسفانه، یک جانشین ایستا به راحتی توسط نمونه‌های تقلبی فریب می‌خورد.
برای بهبود عملکرد چارچوب MetricGAN، برخی تکنیک‌های پیشرفته یادگیری پیشنهاد شده‌اند. در طی این تحقیق، عواملی که به‌طور قابل‌توجهی بر عملکرد یا کارایی آموزش تأثیر می‌گذارند نیز بررسی می‌شوند. بهبود MetricGAN+ عمدتاً از طریق سه تغییر زیر حاصل می‌شود.
\begin{enumerate}
    \item یادگیری امتیازات معیار برای گفتار نویزی
    \item نمونه‌ها از بافر بازپخش تجربه
    \item تابع سیگموید قابل یادگیری برای تخمین ماسک
\end{enumerate}

\begin{figure}[h]

    \centering
    \includegraphics[width=.6\textwidth, keepaspectratio]{images/metricgan.jpg}
    
    \caption{آموزش metricgan}
    \label{fig:metricgan-training}
\end{figure}

در این مطالعه، چندین تکنیک برای بهبود عملکرد چارچوب MetricGAN پیشنهاد کردیم. ما متوجه شدیم که شامل کردن گفتار نویزی برای آموزش تفکیک‌کننده و استفاده از سیگموید قابل یادگیری، مفیدترین تکنیک‌ها هستند. MetricGAN+ ما نتایج پیشرفته‌ای را بر روی مجموعه داده‌های VoiceBank-DEMAND به‌دست می‌آورد و امتیازهای PESQ می‌تواند به ترتیب 0.3 و 0.45 نسبت به MetricGAN و BLSTM (MSE) افزایش یابد.

\section{Diff-TTS}
با وجود اینکه مدل‌های تبدیل متن به گفتار (TTS) عصبی توجه زیادی را جلب کرده و در تولید گفتار شبیه به انسان موفق بوده‌اند، هنوز جای پیشرفت‌هایی برای طبیعی‌تر و کارآمدتر کردن آن‌ها وجود دارد. در این کار، ما یک مدل TTS غیر اتورگرسیو جدید به نام Diff-TTS پیشنهاد می‌کنیم که به تولید گفتار با کیفیت بالا و کارآمدی بالا دست می‌یابد. با توجه به متن، Diff-TTS از یک چارچوب دیفیوژن نویززدایی برای تبدیل سیگنال نویز به طیف‌نگار مل از طریق مراحل زمان دیفیوژن استفاده می‌کند. به‌منظور یادگیری توزیع طیف‌نگار مل با شرط متن، ما یک روش بهینه‌سازی مبتنی بر احتمال برای TTS ارائه می‌دهیم. علاوه بر این، برای افزایش سرعت استنتاج، ما از روش نمونه‌برداری تسریع‌شده استفاده می‌کنیم که به Diff-TTS امکان می‌دهد تا موج‌نگاشت‌های خام را به‌طور بسیار سریع‌تری تولید کند بدون اینکه کیفیت ادراکی به‌طور قابل توجهی کاهش یابد. از طریق آزمایش‌ها، تایید کردیم که Diff-TTS با یک GPU NVIDIA 2080Ti به‌طور 28 برابر سریع‌تر از زمان واقعی تولید می‌کند.
دیف-TTS توزیع نویز را به توزیع مل-اسپکتروگرام متناظر با متن داده شده تبدیل می‌کند. همان‌طور که در شکل 1 نشان داده شده است، مل-اسپکتروگرام به تدریج با نویز گوسی تخریب شده و به متغیرهای نهان تبدیل می‌شود. این فرآیند، فرآیند انتشار نامیده می‌شود.

\begin{figure}[h]

    \centering
    \includegraphics[width=.6\textwidth, keepaspectratio]{images/diffusion.jpg}
    
    \caption{فرایند diffusion}
    \label{fig:diffusion}
\end{figure}

فرض کنید x1, . . . , xT یک دنباله از متغیرها با ابعاد یکسان باشد که در آن t = 0, 1, . . . , T شاخصی برای مراحل زمانی انتشار است. سپس، فرآیند انتشار مل-اسپکتروگرام x0 را از طریق یک زنجیره انتقال‌های مارکوفی به نویز گوسی xT تبدیل می‌کند. هر مرحله انتقال با یک برنامه واریانس β1، β2، ...، βT از پیش تعیین شده است. به طور خاص، هر تبدیل مطابق با احتمال انتقال مارکوفی q(xt|xt−1, c) انجام می‌شود که مستقل از متن c فرض می‌شود و به صورت زیر تعریف می‌شود:

\begin{figure}[h]

    \centering
    \includegraphics[width=.6\textwidth, keepaspectratio]{images/eq1.jpg}
    
    % \caption{فرایند diffusion}
    \label{fig:eq1}
\end{figure}

کل فرآیند انتشارq(x1:T∣x0,c)  یک فرآیند مارکوفی است و می‌تواند به صورت زیر تجزیه شود:

\begin{figure}[h]

    \centering
    \includegraphics[width=.6\textwidth, keepaspectratio]{images/eq2.jpg}
    
    % \caption{فرایند diffusion}
    \label{fig:eq2}
\end{figure}

فرآیند معکوس یک روش تولید مل-اسپکتروگرام است که دقیقاً برخلاف فرآیند انتشار عمل می‌کند. برخلاف فرآیند انتشار، هدف فرآیند معکوس بازیابی یک مل-اسپکتروگرام از نویز گوسی است. فرآیند معکوس به عنوان توزیع شرطی pθ​(x0:T−1​∣xT​,c) تعریف می‌شود و می‌تواند بر اساس خاصیت زنجیره مارکوف به چندین انتقال تجزیه شود:

\begin{figure}[h]

    \centering
    \includegraphics[width=.6\textwidth, keepaspectratio]{images/eq3.jpg}
    
    % \caption{فرایند diffusion}
    \label{fig:eq3}
\end{figure}

دیف-TTS شامل یک رمزگذار متن، رمزگذار مرحله، پیش‌بینی‌کننده مدت زمان، و رمزگشا است. 
شبکه رمزگشا شامل یک پشته از 12 بلوک مقاوم با Conv1D، tanh، sigmoid و کانولوشن‌های 1x1 با 512 کانال مقاوم است [29]. همان‌طور که در شکل 3 نشان داده شده است، تعبیه فونم توسط تنظیم‌کننده طول گسترش می‌یابد. سپس، تعبیه فونم و خروجی رمزگذار مرحله به ورودی پس از لایه Conv1D اضافه می‌شود. لایه Conv1D اندازه هسته‌ای برابر 3 بدون گشادگی دارد. پس از عبور از این بلوک مقاوم، خروجی‌ها قبل از پس-نت جمع می‌شوند. در نهایت، رمزگشا نویز گوسی متناظر با دنباله فونم و مرحله زمانی انتشار را به دست می‌آورد.

\begin{figure}[h]

    \centering
    \includegraphics[width=.6\textwidth, keepaspectratio]{images/diff-tts.jpg}
    
    \caption{معماری diff-tts}
    \label{fig:diff-tts}
\end{figure}

\section{\lr{A Study on Speech Enhancement Based on Diffusion Probabilistic Model}}
مدل‌های احتمالاتی انتشار توانایی فوق‌العاده‌ای در مدل‌سازی تصاویر طبیعی و فرم‌های صوتی خام از طریق فرآیندهای جفت‌شده انتشار و معکوس نشان داده‌اند. ویژگی منحصر به فرد فرآیند معکوس (یعنی حذف سیگنال‌های غیرهدف از نویز گوسی و سیگنال‌های نویزی) می‌تواند برای بازیابی سیگنال‌های تمیز استفاده شود. بر اساس این ویژگی، ما مدل بهبود گفتار مبتنی بر مدل احتمالاتی انتشار (DiffuSE) را پیشنهاد می‌کنیم که هدف آن بازیابی سیگنال‌های گفتاری تمیز از سیگنال‌های نویزی است. معماری اساسی مدل DiffuSE پیشنهادی مشابه معماری DiffWave است—مدل تولید فرم صوتی با کیفیت بالا که هزینه محاسباتی و ردپای نسبتاً پایینی دارد. برای دستیابی به عملکرد بهتر در بهبود، ما فرآیند معکوس پیشرفته‌ای طراحی کردیم که به آن فرآیند معکوس حمایتی گفته می‌شود و در هر مرحله زمانی، گفتار نویزی را به گفتار پیش‌بینی‌شده اضافه می‌کند. نتایج تجربی نشان می‌دهد که DiffuSE عملکردی معادل با مدل‌های تولید صوت مرتبط در وظیفه SE مجموعه داده Voice Bank استاندارد شده دارد. علاوه بر این، نسبت به برنامه نمونه‌برداری کامل معمولاً پیشنهاد شده، فرآیند معکوس حمایتی پیشنهادی به ویژه سرعت نمونه‌برداری سریع را بهبود بخشیده و با انجام چندین مرحله نتایج بهبود بهتری نسبت به فرآیند استنتاج کامل سنتی ارائه می‌دهد.
در مدل پیشنهادی DiffuSE، ما یک فرآیند معکوس حمایتی جدید را استخراج می‌کنیم تا جایگزین فرآیند معکوس اصلی شود و سیگنال‌های نویز را به طور مؤثرتری از ورودی نویزی حذف کنیم.

\subsection{الف. فرآیند معکوس حمایتی}
در مدل احتمالی انتشار اصلی، نویز گوسی در فرآیند معکوس اعمال می‌شود. از آنجا که سیگنال گفتار تمیز در طول فرآیند معکوس دیده نمی‌شود، سیگنال گفتار محاسبه شده  xt​ ممکن است در طول فرآیند معکوس از مرحله T تا t+1 تحریف شود. برای حل این مشکل، ما فرآیند معکوس حمایتی را پیشنهاد دادیم، که فرآیند نمونه‌برداری را از سیگنال گفتار نویزی y آغاز می‌کند و y را در هر مرحله معکوس ترکیب می‌کند در حالی که سیگنال گوسی اضافی را کاهش می‌دهد.
شکل 2 ساختار مدل DiffuSE را نشان می‌دهد. همانند DiffWave، تنظیم‌کننده در DiffuSE هدفش حفظ شباهت سیگنال خروجی به سیگنال گفتار هدف است، که اجازه می‌دهد نویز و گفتار تمیز را از داده‌های مخلوط جدا کند. 

\begin{figure}[h]

    \centering
    \includegraphics[width=.6\textwidth, keepaspectratio]{images/diffuSE.jpg}
    
    \caption{معماری diffuSE}
    \label{fig:diffuSE}
\end{figure}

برای تولید سیگنال‌های گفتار با کیفیت بالا، مدل DiffuSE را با ویژگی‌های Mel-spectral تمیز پیش‌آموزی کردیم. در DiffWave، اطلاعات شرطی مستقیماً از گفتار تمیز گرفته می‌شود که به مدل اجازه می‌دهد تا گفتار تمیز و نویز را از سیگنال‌های مخلوط جدا کند. پس از پیش‌آموزش، تنظیم‌کننده را از ویژگی‌های Mel-spectral تمیز به ویژگی‌های spectral نویزی تغییر دادیم، پارامترهای رمزگذار تنظیم‌کننده را مجدداً تنظیم کردیم و دیگر پارامترها را برای آموزش SE حفظ کردیم. در نهایت از نمونه‌برداری سریع استفاده شد تا تعداد مراحل حذف نویز کاهش پیدا کند.


\end{document}
