\documentclass[fleqn]{report}
\usepackage[utf8]{inputenc}
\usepackage{graphicx}
\usepackage{fancyhdr}
\usepackage{amsmath}
\usepackage{xcolor}
\usepackage{subfigure}
\usepackage{pdfpages}
\usepackage{hyperref}
\usepackage{xepersian}
%In order to write persian in latex, with "texlive" distro installed, two additional packages are needed: texlive-xetex & texlive-lang-arabic

\settextfont[BoldFont={XB Zar bold.ttf}]{XB Zar.ttf}

\begin{document}

%Title Page
\begin{titlepage}
    \begin{center}
        \includegraphics{sharif.png}\\
        \vspace{1cm}    \textbf{\LARGE   \lr{Speech Enhancement and Denoising}}\\
        \vspace{1cm}  \textbf{\Large  پروژه پایانی درس پردازش گفتار}\\
        \vspace{1cm}    \textbf{\large  دانشکده مهندسی کامپیوتر}\\
        \vspace{0.5cm}  \textbf{\large  استاد: دکتر صامتی}\\
        \vspace{0.5cm}  \textbf{\large  بهار ۱۴۰۳}\\
        \vspace{0.5cm}  \noindent\rule{8cm}{0.4pt}\\
        \vspace{2cm}    \textbf{\large بهار بهزادی‌پور \ \   ۴۰۲۲۰۷۱۵۵}\\
        \vspace{0.5cm}    \textbf{\large مهتا فطرت \ \   ۴۰۲۲۱۲۲۲۵}\\
    \end{center}
\end{titlepage}

\tableofcontents

%Header & Footer
\pagestyle{fancy}
\fancyhf{}
\rhead{\includegraphics[scale=0.1]{sharif.png} پردازش گفتار}
\lhead{Speech Enhancement and Denoising}
\rfoot{Page \thepage}

%Document Body

\chapter{مقدمه}
بهبود کیفیت گفتار و رفع نویز آن همواره مورد توجه بوده‌است و جزء معدود مسائل حوزه‌ی پردازش گفتار است که می‌توان گفت هنوز کاملا حل‌شده تلقی نمی‌شو و جای پیشرفت دارد.
با این حال، مدل‌ها و ابزار‌های بسیاری برای بهبود گفتار زبان انگلیسی وجود دارد که از کیفیت مطلوبی نیز برخوردار هستند.
در این میان، زبان فارسی نسبت له تکنولوژی‌های اخیر در این زمینه جای پیشرفت زیادی دارد.
بررسی‌های اولیه نشان می‌دهد که مدل بهبود گفتاری که برای زبان فارسی مناسب‌سازی شده باشد در حال حاضر وجود ندارد.
در این پروژه ما در تلاش هستیم تا وضعیت ابزار‌های حاضر را برای بهبود گفتار فارسی بررسی کنیم و بتوانیم در راه بهبود آن‌ها گام‌های موثری برداریم.

\section{شرح مسئله}
بهبود گفتار \lr{(Speech Enhancement)}  حذف نویز \lr{(Denoising)} به  فرآیندهایی اطلاق میشود که کیفیت و وضوح سیگنال گفتار را بهبود میبخشند و نویزهای پس زمینه را کاهش میدهند. این مسئله در کاربردهای مختلفی از جمله سیستم های تشخیص گفتار، ارتباطات تلفنی، و کمک به افراد با مشکلات شنوایی اهمیت دارد. در زبان فارسی، به دلیل تفاوت های صوتی و ساختاری با دیگر زبان ها، نیاز به تحقیق و توسعه ویژه ای در این زمینه وجود دارد.

\section{رویکردها}
در مورد مسئله‌ی بهبود گفتار و رفع نویز مانند بسیاری از مسائل دیگر در حوزه‌ی گفتار، دو رویکرد کلی وجود دارد.
اولین رویکرد مربوط به روش‌های 
rule-based می‌باشد.
این روش‌ها عموما مبتنی بر تکنیک‌های پردازش سیگنال می‌باشند و به مشاهده‌ی نمونه‌ی گفتار‌های تمیز و نویزی وابسته نیستند.
رویکرد دوم اما
مبتنی بر شبکه‌های 
عصبی می‌باشد.
در این روش‌ها، با داشتن یک معماری مناسب و تعداد زیادی از نمونه‌های ورودی و خروجی مطلوب، مدل می‌آموزد که تسک بهبود گفتار را انجام دهد.
در این پروژه، ما هر دوی این روش‌ها را بررسی می‌کنیم و راه‌حل‌هایی بر اساس هر یک ارائه می‌دهیم.

\section{چالش‌ها}
همانطور که برای سایر راه‌حل‌های مبتنی بر شبکه‌های عصبی  
چالش داده مطرح است، در این‌جا هم با این مشکل مواجه هستیم.
درواقع تهیه‌ی دادگانی طبیعی متشکل از زوج‌های تمیز و نویزی یک گفتار واحد، امری دشوار و زمان‌بر است. 
به همین جهت غالبا شاهد دیتاست‌هایی هستیم که به صورت اتوماتیک generate شده‌اند.
در ادامه به این موضوع بیشتر پرداخته می‌شود.

\section{اهداف و دستاورد‌های این پروژه}
در این پروژه ما ابتدا روش‌های موجود را برای زبان فارسی ارزیابی کردیم و امکان خاص‌سازی آن‌ها برای زبان فارسی را بررسی نمودیم.
سپس از بین روش‌های موجود، سعی در بهبود برخی از این روش‌ها به کمک تکنیک‌هایی چون fine-tuning داشتیم.
همچنین به عنوان یک جایگزین، ابزاری rule-based برای بهبود گفتار فارسی نیز ارائه دادیم که حتی بدون داده‌های آموزش نیز قابل استفاده است.
لازم به ذکر است که در این پروژه، دو نوع دیتاست نیز برای تسک \lr{Speech Enhancement and Denoising} برای زبان فارسی ارائه می‌شود که \lr{to the best of our knowledge} اولین دیتاست‌های این زبان در این تسک،‌ حتی از نوع ساختگی آن می‌باشد.


\end{document}
